\documentclass{article}
    \usepackage[UTF8]{ctex}
    
    \title{Hello LaTeX with 中文}
    \begin{document}
    \maketitle
    \section{图像卷积与信号卷积对照理解}
    本文主要解释如何理解图像卷积与信号卷积之间的关系
      \subsection{Hello 卷积}
      图像在进行{卷积}操作时,通常都是由滤波器“滑过”图像
      
      \subsubsection{回顾}
      典型的滤波器在每一次滑动时,滤波器都对它本身覆盖的图像区域的像素值进行加权求和。
    
      \subsubsection{理解}
      回顾信号中离散卷积和公式
    
      $$y\left[ n \right] = \sum\limits_{k =  - \infty }^{ + \infty } {x\left[ k \right]h\left[ {n - k} \right]}$$
    
      该公式与实际的图像卷积过程等效(暂且将图像想象成一维的数据),具体体现在以下几点:
    
      ${x\left[ k \right]}$当于原始图像,${h\left[ {n - k} \right]}$相当于滑动中的滤波器,$y\left[ n \right]$相当于输出图像。
    
    \subsection{Hello 卷积2}
  
    \end{document}